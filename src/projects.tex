%----------------------------------------------------------------------------------------
%	SELECTED PROJECTS
%----------------------------------------------------------------------------------------

\roottitle{Selected Projects} % Top level section

%%%%%%%%%%%%%%%%%% JUJU %%%%%%%%%%%%%%%%% 
\headedsection % Employer name which can include a hyperlink and location/URL on the right side of the page
{}
{} {
\vspace{-0.4cm}
\headedsubsection % Job title entry for the current employer
{\myhref{https://github.com/juju/juju}{Juju}}
{}
{\small\bodytext{

A large scale distributed orchestration engine for managing cloud workloads on any infrastructure (Kubernetes or otherwise) across various cloud providers (e.g., \acr{AWS}, \acr{GCE}). Juju is written in Go, and has over a million lines of code. I was a core maintainer in a team of 14 engineers. See Canonical above for details of my contribution.

}}
}

%%%%%%%%%%%%%%%%%% TERRAFORM %%%%%%%%%%%%%%%%% 
\headedsection % Employer name which can include a hyperlink and location/URL on the right side of the page
{}
{} {
\vspace{-0.4cm}
\headedsubsection % Job title entry for the current employer
{\myhref{https://github.com/juju/terraform-provider-juju}{Terraform Juju Provider}}
{}
{\small\bodytext{A Terraform provider that enables integration with \myhref{https://github.com/juju/juju}{Juju} while managing Terraform environments. I implemented new resources and features (e.g., \myhref{https://discourse.charmhub.io/t/manual-provisioning-on-aws-using-terraform-juju-provider/10989}{manual provisioning on \acr{AWS}}), migrated the provider from the sdk2 to the provider framework (e.g., \myhref{https://github.com/juju/terraform-provider-juju/pull/264}{sample PR}), and maintained release cadence of new versions. All in Go.}}
}

%%%%%%%%%%%%%%%%%% Pycket %%%%%%%%%%%%%%%%% 

\headedsection % Employer name which can include a hyperlink and location/URL on the right side of the page
{}
{} {
\vspace{-0.4cm}
\headedsubsection % Job title entry for the current employer
{\myhref{http://github.com/pycket/pycket}{Pycket: A meta-tracing \textsc{JIT} compiler for self-hosting Racket}}
{}
{\small\bodytext{

PhD thesis project. Developed and maintained Pycket for over five years, designing the compiler to bootstrap the entire Racket language on a meta-tracing \acr{JIT} backend. Contributed to a new \acr{IR} (linklets, see publications) for a more portable Racket run-time. Built \myhref{https://github.com/cderici/trace-draw}{performance analysis tools} and \myhref{https://github.com/cderici/abstract-machine-interp}{formalisms} to improve meta-tracing efficiency. Implemented run-time optimizations, data structures, and primitives.

}}
}

%%%%%%%%%%%%%%%%%% RAX %%%%%%%%%%%%%%%%% 

\headedsection % Employer name which can include a hyperlink and location/URL on the right side of the page
{}
{} {
\vspace{-0.4cm}
\headedsubsection % Job title entry for the current employer
{\myhref{https://github.com/cderici/rax}{Rax: A full-stack Racket to x86\_64 nanopass compiler}}
{}
{\small\bodytext{Implemented all the passes (e.g., closure conversion, register allocation, code-gen, etc.), along with garbage collection. Developed optimizations, such as inlining, loop-invariant code motion, and proper tail-calls.}}
}

%%%%%%%%%%%%%%%%%% HAZIRCEVAP %%%%%%%%%%%%%%%%% 
\headedsection % Employer name which can include a hyperlink and location/URL on the right side of the page
{}
{} {
\vspace{-0.4cm}
\headedsubsection % Job title entry for the current employer
{HazirCevap (Witty): A closed domain question answering system for high school students}
{}
{\small\bodytext{Government funded large scale question answering system. MSc thesis on \textsc{NLP} and Machine Learning. Led \textsc{R\&D} team (3 faculties, 4 grad students). Developed a Hidden Markov random field model for question analysis, and relevance metrics for information retrieval and response generation (see publications). Full stack in Python, and JavaScript.}}
}
